\input{coverpage}
\documentclass[11.5pt]{article}
\usepackage{titling}
\usepackage{sbc-template}
\usepackage{graphicx,url}

\usepackage[brazil]{babel}
\usepackage[utf8]{inputenc}

\usepackage{multicol}
\usepackage{multirow}
\usepackage{setspace,lipsum}

\sloppy

\date{\today}

%%%%%%%%%%%%%%%%%%%%%%%%%%%%%%%%%%%%%%%%%%%%%%%%%%%%%%%%%%%%%%%%%%%%%%%%%%%%%%%%%%%%%%%%%%%%%%%%%%%%

\title{
    Estudo de Caso da Correlação Entre Cobertura de Código e Falhas Reportadas em um Sistema
    Operacional
}

\local{Porto Alegre}

\author{George Redivo Pinto}

\preambulo{Trabalho apresentado à disciplina Engenharia de Software Aplicada, pelo Curso de
Especialização em Engenharia de Software da Universidade do Vale do Rio dos Sinos - UNISINOS,
ministrada pela professora Josiane Brietzke Porto.}

\address{
  Universidade do Vale do Rio dos Sinos (UNISINOS)\\
  Porto Alegre -- RS -- Brasil
}

\instituicao{
    UNIVERSIDADE DO VALE DO RIO DOS SINOS – UNISINOS

    UNIDADE ACADÊMICA DE PESQUISA E PÓS-GRADUAÇÃO

    ESPECIALIZAÇÃO EM ENGENHARIA DE SOFTWARE
}

\begin{document}

\imprimircapa
\imprimirfolhaderosto

\maketitle

\begin{abstract}

Software testing is an important step in software development cycle. This step aims to ensure that
the code works according to the specification.
Code coverage is a metric that tries to measure the test quality by indicating the percentage of
code that were used during test execution.
This article aims to research the correlation between code coverage rate and software quality,
measured by reported faults.
To do that it was done a case study using the features of an operational system and the results
indicate that there is a correlation between code coverage rate and software quality.

\end{abstract}

\begin{resumo}

O teste de \textit{software} é uma importante etapa no processo de desenvolvimento de \textit{software}. Essa etapa
visa garantir que o \textit{software} funciona de acordo com as especificações.
A cobertura de código é uma métricas que se propõe a medir a qualidade de um teste. Ela consiste em
prover uma taxa percentual que indica a porcentagem de código executadas nos testes.
O presente artigo visa estudar se existe ou não correlação entre a cobertura de código e a
qualidade do \textit{software}, medida em quantidade de falhas reportadas.
Para tal, foi feito um estudo de caso utilizando as funcionalidades de um sistema operacional,
chegando em um resultado que indica a existência de correlação entre cobertura de ódigo e
qualidade de \textit{software}.


\end{resumo}


%%%%%%%%%%%%%%%%%%%%%%%%%%%%%%%%%%%%%%%%%%%%%%%%%%%%%%%%%%%%%%%%%%%%%%%%%%%%%%%%%%%%%%%%%%%%%%%%%%%%
%%%%%%%%%%%%%%%%%%%%%%%%%%%%%%%%%%%%%%%%%%%%%%%%%%%%%%%%%%%%%%%%%%%%%%%%%%%%%%%%%%%%%%%%%%%%%%%%%%%%
%%%%%%%%%%%%%%%%%%%%%%%%%%%%%%%%%%%%%%%%%%%%%%%%%%%%%%%%%%%%%%%%%%%%%%%%%%%%%%%%%%%%%%%%%%%%%%%%%%%%


\section{Introdução}

A etapa de teste de \textit{software} é uma etapa importante no processo de desenvolvimento de
\textit{software} e está cada vez mais presente no cotidiano das equipes de desenvolvimento de
\textit{software}.
Essa etapa visa garantir a funcionalidade de um dado \textit{software}, podendo ser feita utilizando
diversos métodos e em diversos níveis, de acordo com a finalidade do teste.
Contudo, o desenvolvimento de testes implica em um impacto financeiro no projeto, uma vez que muitas
horas e profissionais são demandados para a escrita de testes.

A cobertura de código é uma métrica que indica a porcentagem de código que está sendo coberta pelos
testes executados.
Essa métrica ajuda a medir a qualidade dos casos de teste em questão e também pode servir como uma
métrica que indica quando podemos parar de escrever testes, ou seja, quando já temos uma quantidade
suficiente de testes para cobrir um determinado código, com a finalidade de redução de custos.
Contudo, a decisão pela parada da escrita de testes pode impactar na qualidade do \textit{software}
final, uma vez que alguns casos de uso podem ficar descoberto pelos testes.
Por esse motivo é imprescindível que tal decisão seja tomada baseada em dados sólidos, a fim de
evitar esforços excessivos no desenvolvimento de teste, porém garantindo a qualidade do
\textit{software}.
Segundo \cite{coverageAtGoogle}, a adoção da análise de cobertura de código vem crescendo ao longo
dos anos e tem uma boa avaliação de seus usuários quanto à sua efetividade, o que reafirma a
necessidade de que se tenha um bom conhecimento sobre a efetividade de tal métrica para medir
qualidade de teste e de \textit{software}.

Diversos estudos tentam verificar se existe, de fato, uma correlação entre cobertura de testes e
qualidade de \textit{software}, tais como \cite{coverageMetaAnalysis}, \cite{unitTestedCrash} e
\cite{coverageLargeScaleStudy}.
Os resultados obtidos por esses estudos são controversos entre si.
\cite{coverageMetaAnalysis} traz uma revisão sistemática de diversos artigos publicados sobre o
assunto e aponta algumas possíveis vulnerabilidades metodológicas nos artigos estudados, o que
sugere que mais estudos podem ser feitos para tentar delinear melhor a relação entre cobertura de
código e qualidade de \textit{software}.

Nesse sentido, o presente artigo tem por objetivo verificar se existe uma correlação entre cobertura
de código e qualidade de \textit{software}, que serão medidas a com métricas de cobertura de linhas
e número de falhar reportadas, respectivamente.
Para tal, o estudo tem o objetivo de fazer um estudo de caso, conforme descrito em
\cite{metodosPesquisa}, em um projeto de sistema operacional embarcado desenvolvido por uma empresa
brasileira, observando as métricas supracitadas e usando métodos estatísticos para identificar a
correlação entre cobertura de código e qualidade de \textit{software}.
Nesse contexto, as seguintes etapas serão desenvolvidas:

\begin{itemize}
    \item Coletar dados de cobertura de código (por repositório) e falhas reportadas do projeto (por
          funcionalidade);

    \item Sanitizar dados, excluindo dados inválidos para a pesquisa;

    \item Relacionar os repositórios com as funcionalidades;

    \item Fazer análise estatística dos dados coletados.
\end{itemize}

Os resultados da presente pesquisa contribuem para uma maior clareza sobre a correlação entre
cobertura de código e qualidade de \textit{software}, o que pode ajudar a balizar decisões de
alocação de recursos humanos na atividade de escrita de testes de \textit{software}.
Além disso, os resultados obtidos podem ser usados como base acadêmica na área de Engenharia de
Software para um melhor entendimento dessa correlação, além de produzir insumos para estudos futuros
nesse mesmo contexto.

O artigo está dividido seis sessões, são elas:
\begin{enumerate}
    \item Introdução: tem por objetivo fornecer uma visão geral da presente pesquisa, além da
          motivação e etapas desenvolvidas;

    \item Fundamentação Teórica e Ferramentas: apresenta um embasamento teórico dos assuntos
          abortados no artigo, além de listar as principais ferramentas utilizadas na coleta,
          triagem e análise dos dados;

    \item Trabalhos Relacionados: revisita alguns trabalhos recentes relacionados ao tema da
          presente pesquisa;

    \item Metodologia de Pesquisa: apresenta a metodologia utilizada no desenvolvimento do estudo
          proposto;

    \item Resultados: detalha os resultados obtidos da análise dos dados;

    \item Considerações Finais: faz um resumo dos resultados, vulnerabilidades da pesquisa, além de
          indicar possíveis estudos complementares.
\end{enumerate}



%%%%%%%%%%%%%%%%%%%%%%%%%%%%%%%%%%%%%%%%%%%%%%%%%%%%%%%%%%%%%%%%%%%%%%%%%%%%%%%%%%%%%%%%%%%%%%%%%%%%
%%%%%%%%%%%%%%%%%%%%%%%%%%%%%%%%%%%%%%%%%%%%%%%%%%%%%%%%%%%%%%%%%%%%%%%%%%%%%%%%%%%%%%%%%%%%%%%%%%%%
%%%%%%%%%%%%%%%%%%%%%%%%%%%%%%%%%%%%%%%%%%%%%%%%%%%%%%%%%%%%%%%%%%%%%%%%%%%%%%%%%%%%%%%%%%%%%%%%%%%%


\section{Fundamentação Teórica e Ferramentas}
Blablabla

%%%%%%%%%%%%%%%%%%%%%%%%%%%%%%%%%%%%%%%%%%%%%%%%%%%%%%%%%%%%%%%%%%%%%%%%%%%%%%%%%%%%%%%%%%%%%%%%%%%%

\subsection{Testes de Software}
Blablabla

%%%%%%%%%%%%%%%%%%%%%%%%%%%%%%%%%%%%%%%%%%%%%%%%%%%%%%%%%%%%%%%%%%%%%%%%%%%%%%%%%%%%%%%%%%%%%%%%%%%%

\subsection{Cobertura de Código}
Blablabla

%%%%%%%%%%%%%%%%%%%%%%%%%%%%%%%%%%%%%%%%%%%%%%%%%%%%%%%%%%%%%%%%%%%%%%%%%%%%%%%%%%%%%%%%%%%%%%%%%%%%

\subsection{Integração Contínua}
Blablabla

%%%%%%%%%%%%%%%%%%%%%%%%%%%%%%%%%%%%%%%%%%%%%%%%%%%%%%%%%%%%%%%%%%%%%%%%%%%%%%%%%%%%%%%%%%%%%%%%%%%%

\subsection{Ferramentas}
Blablabla

\subsubsection{Bugzilla}
Blablabla

\subsubsection{GoogleTest}
Blablabla

\subsubsection{Gcov}
Blablabla

\subsubsection{Máquinas Virtuais}
Blablabla

\subsubsection{Integração contínua com Gerrit e Jenkins}
Blablabla


%%%%%%%%%%%%%%%%%%%%%%%%%%%%%%%%%%%%%%%%%%%%%%%%%%%%%%%%%%%%%%%%%%%%%%%%%%%%%%%%%%%%%%%%%%%%%%%%%%%%
%%%%%%%%%%%%%%%%%%%%%%%%%%%%%%%%%%%%%%%%%%%%%%%%%%%%%%%%%%%%%%%%%%%%%%%%%%%%%%%%%%%%%%%%%%%%%%%%%%%%
%%%%%%%%%%%%%%%%%%%%%%%%%%%%%%%%%%%%%%%%%%%%%%%%%%%%%%%%%%%%%%%%%%%%%%%%%%%%%%%%%%%%%%%%%%%%%%%%%%%%


\section{Trabalhos Relacionados}
Blablabla


%%%%%%%%%%%%%%%%%%%%%%%%%%%%%%%%%%%%%%%%%%%%%%%%%%%%%%%%%%%%%%%%%%%%%%%%%%%%%%%%%%%%%%%%%%%%%%%%%%%%
%%%%%%%%%%%%%%%%%%%%%%%%%%%%%%%%%%%%%%%%%%%%%%%%%%%%%%%%%%%%%%%%%%%%%%%%%%%%%%%%%%%%%%%%%%%%%%%%%%%%
%%%%%%%%%%%%%%%%%%%%%%%%%%%%%%%%%%%%%%%%%%%%%%%%%%%%%%%%%%%%%%%%%%%%%%%%%%%%%%%%%%%%%%%%%%%%%%%%%%%%


\section{Metodologia de Pesquisa}
Blablabla

%%%%%%%%%%%%%%%%%%%%%%%%%%%%%%%%%%%%%%%%%%%%%%%%%%%%%%%%%%%%%%%%%%%%%%%%%%%%%%%%%%%%%%%%%%%%%%%%%%%%

\subsection{Delineamento de Pesquisa}
Blablabla

%%%%%%%%%%%%%%%%%%%%%%%%%%%%%%%%%%%%%%%%%%%%%%%%%%%%%%%%%%%%%%%%%%%%%%%%%%%%%%%%%%%%%%%%%%%%%%%%%%%%

\subsection{Unidade de Análise}
Blablabla

%%%%%%%%%%%%%%%%%%%%%%%%%%%%%%%%%%%%%%%%%%%%%%%%%%%%%%%%%%%%%%%%%%%%%%%%%%%%%%%%%%%%%%%%%%%%%%%%%%%%

\subsection{Coleta de Dados}
Blablabla

\subsubsection{Falhas Reportadas}
Blablabla

\subsubsection{Cobertura de Código}
Blablabla

%%%%%%%%%%%%%%%%%%%%%%%%%%%%%%%%%%%%%%%%%%%%%%%%%%%%%%%%%%%%%%%%%%%%%%%%%%%%%%%%%%%%%%%%%%%%%%%%%%%%

\subsection{Análise e Triagem de dados}
Blablabla

%%%%%%%%%%%%%%%%%%%%%%%%%%%%%%%%%%%%%%%%%%%%%%%%%%%%%%%%%%%%%%%%%%%%%%%%%%%%%%%%%%%%%%%%%%%%%%%%%%%%

\subsection{Limitações Apresentadas}
Blablabla


%%%%%%%%%%%%%%%%%%%%%%%%%%%%%%%%%%%%%%%%%%%%%%%%%%%%%%%%%%%%%%%%%%%%%%%%%%%%%%%%%%%%%%%%%%%%%%%%%%%%
%%%%%%%%%%%%%%%%%%%%%%%%%%%%%%%%%%%%%%%%%%%%%%%%%%%%%%%%%%%%%%%%%%%%%%%%%%%%%%%%%%%%%%%%%%%%%%%%%%%%
%%%%%%%%%%%%%%%%%%%%%%%%%%%%%%%%%%%%%%%%%%%%%%%%%%%%%%%%%%%%%%%%%%%%%%%%%%%%%%%%%%%%%%%%%%%%%%%%%%%%


\section{Resultados}
Blablabla

%%%%%%%%%%%%%%%%%%%%%%%%%%%%%%%%%%%%%%%%%%%%%%%%%%%%%%%%%%%%%%%%%%%%%%%%%%%%%%%%%%%%%%%%%%%%%%%%%%%%

\subsection{Falhas Reportadas}
Blablabla

%%%%%%%%%%%%%%%%%%%%%%%%%%%%%%%%%%%%%%%%%%%%%%%%%%%%%%%%%%%%%%%%%%%%%%%%%%%%%%%%%%%%%%%%%%%%%%%%%%%%

\subsection{Cobertura de Código}
Blablabla

%%%%%%%%%%%%%%%%%%%%%%%%%%%%%%%%%%%%%%%%%%%%%%%%%%%%%%%%%%%%%%%%%%%%%%%%%%%%%%%%%%%%%%%%%%%%%%%%%%%%

\subsection{Correlação entre Cobertura de Código e Falhas Reportadas}
Blablabla


%%%%%%%%%%%%%%%%%%%%%%%%%%%%%%%%%%%%%%%%%%%%%%%%%%%%%%%%%%%%%%%%%%%%%%%%%%%%%%%%%%%%%%%%%%%%%%%%%%%%
%%%%%%%%%%%%%%%%%%%%%%%%%%%%%%%%%%%%%%%%%%%%%%%%%%%%%%%%%%%%%%%%%%%%%%%%%%%%%%%%%%%%%%%%%%%%%%%%%%%%
%%%%%%%%%%%%%%%%%%%%%%%%%%%%%%%%%%%%%%%%%%%%%%%%%%%%%%%%%%%%%%%%%%%%%%%%%%%%%%%%%%%%%%%%%%%%%%%%%%%%


\section{Considerações Finais}
Blablabla
6. Considerações Finais


%%%%%%%%%%%%%%%%%%%%%%%%%%%%%%%%%%%%%%%%%%%%%%%%%%%%%%%%%%%%%%%%%%%%%%%%%%%%%%%%%%%%%%%%%%%%%%%%%%%%
%%%%%%%%%%%%%%%%%%%%%%%%%%%%%%%%%%%%%%%%%%%%%%%%%%%%%%%%%%%%%%%%%%%%%%%%%%%%%%%%%%%%%%%%%%%%%%%%%%%%
%%%%%%%%%%%%%%%%%%%%%%%%%%%%%%%%%%%%%%%%%%%%%%%%%%%%%%%%%%%%%%%%%%%%%%%%%%%%%%%%%%%%%%%%%%%%%%%%%%%%


As referências bibliográficas devem ser únicas e uniformes. Recomendamos que as referências aos nomes dos autores estejam entre chaves, ex.: \cite{knuth:84}.
As referências bibliográficas devem ser únicas e uniformes. Recomendamos que as referências aos nomes dos autores estejam entre chaves, ex.: \cite{boulic:91}.
As referências bibliográficas devem ser únicas e uniformes. Recomendamos que as referências aos nomes dos autores estejam entre chaves, ex.: \cite{smith:99}.
As referências devem ser listadas usando fonte de 12 pontos, com 6 pontos de espaço entre cada uma. A primeira linha de cada referência não deve ser recuada, enquanto as linhas subsequentes devem possuir recuo de 0,5cm.

\bibliographystyle{sbc}
\bibliography{article}

\end{document}
