\input{coverpage}
\documentclass[11.5pt]{article}
\usepackage{titling}
\usepackage{sbc-template}
\usepackage{graphicx,url}

\usepackage[brazil]{babel}
\usepackage[utf8]{inputenc}

\usepackage{multicol}
\usepackage{multirow}
\usepackage{setspace,lipsum}

\sloppy

\date{\today}

%%%%%%%%%%%%%%%%%%%%%%%%%%%%%%%%%%%%%%%%%%%%%%%%%%%%%%%%%%%%%%%%%%%%%%%%%%%%%%%%%%%%%%%%%%%%%%%%%%%%

\title{
    Estudo de Caso da Correlação Entre Cobertura de Código e Falhas Reportadas em um Sistema
    Operacional
}

\local{Porto Alegre}

\author{George Redivo Pinto}

\preambulo{Trabalho apresentado à disciplina Engenharia de Software Aplicada, pelo Curso de
Especialização em Engenharia de Software da Universidade do Vale do Rio dos Sinos - UNISINOS,
ministrada pela professora Josiane Brietzke Porto.}

\address{
  Universidade do Vale do Rio dos Sinos (UNISINOS)\\
  Porto Alegre -- RS -- Brasil
}

\instituicao{
    UNIVERSIDADE DO VALE DO RIO DOS SINOS – UNISINOS

    UNIDADE ACADÊMICA DE PESQUISA E PÓS-GRADUAÇÃO

    ESPECIALIZAÇÃO EM ENGENHARIA DE SOFTWARE
}

\begin{document}

\imprimircapa
\imprimirfolhaderosto

\maketitle

\begin{abstract}

Software testing is an important step in software development cycle. This step aims to ensure that
the code works according to the specification.
Code coverage is a metric that tries to measure the test quality by indicating the percentage of
code that were used during test execution.
This paper aims to research the correlation between code coverage rate and software quality,
measured by reported faults.
To do that it was done a case study using the features of an operational system and the results
indicate that there is a correlation between code coverage rate and software quality.

\end{abstract}

\begin{resumo}

O teste de software é uma importante etapa no processo de desenvolvimento de software. Essa etapa
visa garantir que o software funciona de acordo com as especificações.
A cobertura de código é uma métricas que se propõe a medir a qualidade de um teste. Ela consiste em
prover uma taxa percentual que indica a porcentagem de código executadas nos testes.
O presente artigo visa estudar se existe ou não correlação entre a cobertura de código e a
qualidade do software, medida em quantidade de falhas reportadas.
Para tal, foi feito um estudo de caso utilizando as funcionalidades de um sistema operacional,
chegando em um resultado que indica a existência de correlação entre cobertura de ódigo e
qualidade de software.


\end{resumo}


%%%%%%%%%%%%%%%%%%%%%%%%%%%%%%%%%%%%%%%%%%%%%%%%%%%%%%%%%%%%%%%%%%%%%%%%%%%%%%%%%%%%%%%%%%%%%%%%%%%%
%%%%%%%%%%%%%%%%%%%%%%%%%%%%%%%%%%%%%%%%%%%%%%%%%%%%%%%%%%%%%%%%%%%%%%%%%%%%%%%%%%%%%%%%%%%%%%%%%%%%
%%%%%%%%%%%%%%%%%%%%%%%%%%%%%%%%%%%%%%%%%%%%%%%%%%%%%%%%%%%%%%%%%%%%%%%%%%%%%%%%%%%%%%%%%%%%%%%%%%%%


\section{Introdução}
Blablabla


%%%%%%%%%%%%%%%%%%%%%%%%%%%%%%%%%%%%%%%%%%%%%%%%%%%%%%%%%%%%%%%%%%%%%%%%%%%%%%%%%%%%%%%%%%%%%%%%%%%%
%%%%%%%%%%%%%%%%%%%%%%%%%%%%%%%%%%%%%%%%%%%%%%%%%%%%%%%%%%%%%%%%%%%%%%%%%%%%%%%%%%%%%%%%%%%%%%%%%%%%
%%%%%%%%%%%%%%%%%%%%%%%%%%%%%%%%%%%%%%%%%%%%%%%%%%%%%%%%%%%%%%%%%%%%%%%%%%%%%%%%%%%%%%%%%%%%%%%%%%%%


\section{Fundamentação Teórica}
Blablabla

%%%%%%%%%%%%%%%%%%%%%%%%%%%%%%%%%%%%%%%%%%%%%%%%%%%%%%%%%%%%%%%%%%%%%%%%%%%%%%%%%%%%%%%%%%%%%%%%%%%%

\subsection{Testes de Software}
Blablabla

%%%%%%%%%%%%%%%%%%%%%%%%%%%%%%%%%%%%%%%%%%%%%%%%%%%%%%%%%%%%%%%%%%%%%%%%%%%%%%%%%%%%%%%%%%%%%%%%%%%%

\subsection{Cobertura de Código}
Blablabla

%%%%%%%%%%%%%%%%%%%%%%%%%%%%%%%%%%%%%%%%%%%%%%%%%%%%%%%%%%%%%%%%%%%%%%%%%%%%%%%%%%%%%%%%%%%%%%%%%%%%

\subsection{Test Driven Development -- TDD}
Blablabla

%%%%%%%%%%%%%%%%%%%%%%%%%%%%%%%%%%%%%%%%%%%%%%%%%%%%%%%%%%%%%%%%%%%%%%%%%%%%%%%%%%%%%%%%%%%%%%%%%%%%

\subsection{Integração Contínua}
Blablabla

%%%%%%%%%%%%%%%%%%%%%%%%%%%%%%%%%%%%%%%%%%%%%%%%%%%%%%%%%%%%%%%%%%%%%%%%%%%%%%%%%%%%%%%%%%%%%%%%%%%%

\subsection{Ferramentas}
Blablabla

\subsubsection{Bugzilla}
Blablabla

\subsubsection{GoogleTest}
Blablabla

\subsubsection{Gcov}
Blablabla

\subsubsection{Máquinas Virtuais}
Blablabla

\subsubsection{Integração contínua com Gerrit e Jenkins}
Blablabla


%%%%%%%%%%%%%%%%%%%%%%%%%%%%%%%%%%%%%%%%%%%%%%%%%%%%%%%%%%%%%%%%%%%%%%%%%%%%%%%%%%%%%%%%%%%%%%%%%%%%
%%%%%%%%%%%%%%%%%%%%%%%%%%%%%%%%%%%%%%%%%%%%%%%%%%%%%%%%%%%%%%%%%%%%%%%%%%%%%%%%%%%%%%%%%%%%%%%%%%%%
%%%%%%%%%%%%%%%%%%%%%%%%%%%%%%%%%%%%%%%%%%%%%%%%%%%%%%%%%%%%%%%%%%%%%%%%%%%%%%%%%%%%%%%%%%%%%%%%%%%%


\section{Trabalhos Relacionados}
Blablabla


%%%%%%%%%%%%%%%%%%%%%%%%%%%%%%%%%%%%%%%%%%%%%%%%%%%%%%%%%%%%%%%%%%%%%%%%%%%%%%%%%%%%%%%%%%%%%%%%%%%%
%%%%%%%%%%%%%%%%%%%%%%%%%%%%%%%%%%%%%%%%%%%%%%%%%%%%%%%%%%%%%%%%%%%%%%%%%%%%%%%%%%%%%%%%%%%%%%%%%%%%
%%%%%%%%%%%%%%%%%%%%%%%%%%%%%%%%%%%%%%%%%%%%%%%%%%%%%%%%%%%%%%%%%%%%%%%%%%%%%%%%%%%%%%%%%%%%%%%%%%%%


\section{Metodologia de Pesquisa}
Blablabla

%%%%%%%%%%%%%%%%%%%%%%%%%%%%%%%%%%%%%%%%%%%%%%%%%%%%%%%%%%%%%%%%%%%%%%%%%%%%%%%%%%%%%%%%%%%%%%%%%%%%

\subsection{Delineamento de Pesquisa}
Blablabla

%%%%%%%%%%%%%%%%%%%%%%%%%%%%%%%%%%%%%%%%%%%%%%%%%%%%%%%%%%%%%%%%%%%%%%%%%%%%%%%%%%%%%%%%%%%%%%%%%%%%

\subsection{Unidade de Análise}
Blablabla

%%%%%%%%%%%%%%%%%%%%%%%%%%%%%%%%%%%%%%%%%%%%%%%%%%%%%%%%%%%%%%%%%%%%%%%%%%%%%%%%%%%%%%%%%%%%%%%%%%%%

\subsection{Coleta de Dados}
Blablabla

\subsubsection{Falhas Reportadas}
Blablabla

\subsubsection{Cobertura de Código}
Blablabla

%%%%%%%%%%%%%%%%%%%%%%%%%%%%%%%%%%%%%%%%%%%%%%%%%%%%%%%%%%%%%%%%%%%%%%%%%%%%%%%%%%%%%%%%%%%%%%%%%%%%

\subsection{Análise e Triagem de dados}
Blablabla

%%%%%%%%%%%%%%%%%%%%%%%%%%%%%%%%%%%%%%%%%%%%%%%%%%%%%%%%%%%%%%%%%%%%%%%%%%%%%%%%%%%%%%%%%%%%%%%%%%%%

\subsection{Limitações Apresentadas}
Blablabla


%%%%%%%%%%%%%%%%%%%%%%%%%%%%%%%%%%%%%%%%%%%%%%%%%%%%%%%%%%%%%%%%%%%%%%%%%%%%%%%%%%%%%%%%%%%%%%%%%%%%
%%%%%%%%%%%%%%%%%%%%%%%%%%%%%%%%%%%%%%%%%%%%%%%%%%%%%%%%%%%%%%%%%%%%%%%%%%%%%%%%%%%%%%%%%%%%%%%%%%%%
%%%%%%%%%%%%%%%%%%%%%%%%%%%%%%%%%%%%%%%%%%%%%%%%%%%%%%%%%%%%%%%%%%%%%%%%%%%%%%%%%%%%%%%%%%%%%%%%%%%%


\section{Resultados}
Blablabla

%%%%%%%%%%%%%%%%%%%%%%%%%%%%%%%%%%%%%%%%%%%%%%%%%%%%%%%%%%%%%%%%%%%%%%%%%%%%%%%%%%%%%%%%%%%%%%%%%%%%

\subsection{Falhas Reportadas}
Blablabla

%%%%%%%%%%%%%%%%%%%%%%%%%%%%%%%%%%%%%%%%%%%%%%%%%%%%%%%%%%%%%%%%%%%%%%%%%%%%%%%%%%%%%%%%%%%%%%%%%%%%

\subsection{Cobertura de Código}
Blablabla

%%%%%%%%%%%%%%%%%%%%%%%%%%%%%%%%%%%%%%%%%%%%%%%%%%%%%%%%%%%%%%%%%%%%%%%%%%%%%%%%%%%%%%%%%%%%%%%%%%%%

\subsection{Correlação entre Cobertura de Código e Falhas Reportadas}
Blablabla


%%%%%%%%%%%%%%%%%%%%%%%%%%%%%%%%%%%%%%%%%%%%%%%%%%%%%%%%%%%%%%%%%%%%%%%%%%%%%%%%%%%%%%%%%%%%%%%%%%%%
%%%%%%%%%%%%%%%%%%%%%%%%%%%%%%%%%%%%%%%%%%%%%%%%%%%%%%%%%%%%%%%%%%%%%%%%%%%%%%%%%%%%%%%%%%%%%%%%%%%%
%%%%%%%%%%%%%%%%%%%%%%%%%%%%%%%%%%%%%%%%%%%%%%%%%%%%%%%%%%%%%%%%%%%%%%%%%%%%%%%%%%%%%%%%%%%%%%%%%%%%


\section{Considerações Finais}
Blablabla
6. Considerações Finais


%%%%%%%%%%%%%%%%%%%%%%%%%%%%%%%%%%%%%%%%%%%%%%%%%%%%%%%%%%%%%%%%%%%%%%%%%%%%%%%%%%%%%%%%%%%%%%%%%%%%
%%%%%%%%%%%%%%%%%%%%%%%%%%%%%%%%%%%%%%%%%%%%%%%%%%%%%%%%%%%%%%%%%%%%%%%%%%%%%%%%%%%%%%%%%%%%%%%%%%%%
%%%%%%%%%%%%%%%%%%%%%%%%%%%%%%%%%%%%%%%%%%%%%%%%%%%%%%%%%%%%%%%%%%%%%%%%%%%%%%%%%%%%%%%%%%%%%%%%%%%%


As referências bibliográficas devem ser únicas e uniformes. Recomendamos que as referências aos nomes dos autores estejam entre chaves, ex.: \cite{knuth:84}.
As referências bibliográficas devem ser únicas e uniformes. Recomendamos que as referências aos nomes dos autores estejam entre chaves, ex.: \cite{boulic:91}.
As referências bibliográficas devem ser únicas e uniformes. Recomendamos que as referências aos nomes dos autores estejam entre chaves, ex.: \cite{smith:99}.
As referências devem ser listadas usando fonte de 12 pontos, com 6 pontos de espaço entre cada uma. A primeira linha de cada referência não deve ser recuada, enquanto as linhas subsequentes devem possuir recuo de 0,5cm.

\bibliographystyle{sbc}
\bibliography{article}

\end{document}
