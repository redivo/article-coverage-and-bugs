
% Comandos de dados - titulo do documento
\newcommand{\titulo}[1]{\title{#1}}
\newcommand{\imprimirtitulo}{\thetitle}

% Comandos de dados - autor (use \and para múltiplos autores)
\newcommand{\autor}[1]{\author{#1}}
\newcommand{\imprimirautor}{\theauthor}

% Comandos de dados - data
\let\olddate\date
\renewcommand{\date}[1]{\AtBeginDocument{\olddate{#1}}}
\newcommand{\data}[1]{\date{#1}}
\newcommand{\imprimirdata}{
  \center
  \the\year
}

% Comandos de dados - instituição
\providecommand{\imprimirinstituicao}{}
\newcommand{\instituicao}[1]{\renewcommand{\imprimirinstituicao}{#1}}

% Comandos de dados - local
\providecommand{\imprimirlocal}{}
\newcommand{\local}[1]{\renewcommand{\imprimirlocal}{#1}}

% Comandos de dados - preambulo
\providecommand{\imprimirpreambulo}{}
\newcommand{\preambulo}[1]{\renewcommand{\imprimirpreambulo}{#1}}

% Comandos de dados - orientador
\providecommand{\imprimirorientadorRotulo}{}
\providecommand{\imprimirorientador}{}
\newcommand{\orientador}[2][\orientadorname]%
  {\renewcommand{\imprimirorientadorRotulo}{#1}%
   \renewcommand{\imprimirorientador}{#2}}

% Comandos de dados - coorientador
\providecommand{\imprimircoorientadorRotulo}{}
\providecommand{\imprimircoorientador}{}
\newcommand{\coorientador}[2][\coorientadorname]%
  {\renewcommand{\imprimircoorientadorRotulo}{#1}%
   \renewcommand{\imprimircoorientador}{#2}}

% Comandos de dados - tipo de trabalho
\providecommand{\imprimirtipotrabalho}{}
\newcommand{\tipotrabalho}[1]{\renewcommand{\imprimirtipotrabalho}{#1}}


\newcommand{\imprimircapa}{%
  \begin{capa}%
    \center
    \imprimirinstituicao
    \vfill
    \large\imprimirautor

    \vfill
    \begin{center}
      \MakeUppercase{\imprimirtitulo}
    \end{center}
    \vfill

    \large\imprimirlocal

    \large\imprimirdata

    \vspace*{1cm}
  \end{capa}
}

\newenvironment{capa}{\begin{titlingpage}}{\end{titlingpage}\cleardoublepage}

% ---
% Folha de rosto
%   usar \imprimirfolhaderosto* caso deseje imprimir algo no verso da
%   página no caso de estar no modo twoside. Util para imprimir a Ficha
%   Bibliográfica. Porem, se estiver no modo oneside, a versao sem estrela é idêntica.
\newenvironment{folhaderosto}[1][\folhaderostoname]{\clearpage}{\cleardoublepage}
\newenvironment{folhaderosto*}[1][\folhaderostoname]{\clearpage}{\newpage}%

% ---
% Conteudo padrao da Folha de Rosto
\makeatletter
\newcommand{\folhaderostocontent}{
  \begin{center}
    \imprimirinstituicao\vspace*{\fill}

    \large\imprimirautor

    \vspace*{\fill}\vspace*{\fill}
    \begin{center}
      \MakeUppercase{\imprimirtitulo}
    \end{center}
    \vspace*{\fill}

    \hspace{.45\textwidth}
    \begin{minipage}{.5\textwidth}
       \singlespacing
       \imprimirpreambulo
     \end{minipage}
     \vspace*{\fill}

    \large\imprimirlocal
    \par
    \large\imprimirdata
    \vspace*{1cm}

  \end{center}
}
\makeatother

\newcommand{\imprimirfolhaderostostar}[1]{%
  \begin{folhaderosto*}{#1}
     \folhaderostocontent
  \end{folhaderosto*}}

\newcommand{\imprimirfolhaderostonostar}[1]{%
  \begin{folhaderosto}{#1}
     \folhaderostocontent
  \end{folhaderosto}}

\makeatletter
\newcommand{\imprimirfolhaderosto}[1][\folhaderostoname]{%
   \@ifstar
     \imprimirfolhaderostostar
     \imprimirfolhaderostonostar
}
\makeatother

